\documentclass[12pt]{book}
\usepackage[ansinew]{inputenc} 
\usepackage[T1]{fontenc} 
\usepackage[francais]{babel}
\usepackage{amsfonts}
\newcommand{\s}{\begin{center}
*
\end{center}
}


\begin{document}
\title{L'île du changeur}
\maketitle

Isaac se réveilla en sursaut. Le corps inondé de sueur, il tenta de se dégager de la couverture chaude et humide qui le couvrait, mais une douleur sourde s'éveilla dans tout son corps alors qu'il essayait de soulever ses bras affaiblis. Il parvint à tourner légèrement son buste et observa de ses yeux cernés, mi-clos et si enfoncés dans leurs orbites qu'ils se trouvaient au milieu d'une ombre permanente, l'épais grillage en métal situé deux mètres plus bas. Dans un nouvel effort qui lui donna la sensation que tous les muscles et les tendons de son buste se déchiraient brin après brin, chacun claquant dans une étincelle de douleur, Isaac parvint à déplacer son centre de gravité suffisamment vers le bord du lit pour qu'il se sente entraîné vers le bas. Il savoura les quelques secondes pendant lesquelles sont corps glissa hors de la couverture lubrifiée par sa sueur, et tomba lourdement sur le grillage.


\s

Les paupières d'Isaac lui semblaient d'une lourdeur infinie, mais il les souleva lentement. Il se demanda si quelqu'un les avaient enduites de plomb, et immédiatement après s'interrogea sur la possibilité d'introduire des grains de sels sous les paupières d'une personne sans qu'elle se réveille. Un larme se forma au coin de son œil droit et coula jusqu'à ses lèvres. Le goût l'extirpa de sa torpeur. Il avait les yeux ouverts, mais cela ne faisait pas grande différence. Il prit soudain conscience du bruit qui l'entourait et un mal de crâne émergea lentement depuis sont occiput pour venir lentement couler jusqu'à son front. Ses lèvres étaient à nouveau sèches. Il ouvrit délicatement la bouche et sentit les muqueuses se détacher l'une de l'autre. Il extirpa sa langue et sa gorge lui rappela alors son existence avec autant de calme qu'un écorché vif se fait frictionner avec du verre pilé. La sensation du grillage sous lui, incrusté dans sa peau, désintéressa définitivement Isaac de son propre sort. Son renoncement lui parut d'une telle ironie qu'il aurait sourit en refermant les yeux si ses lèvres n'avaient pas été aussi gercées.


\s


Le bruit venait des décélérateurs. La partie la plus difficile d'un voyage interstellaire. Utiliser les vents solaires pour lentement accélérer et se diriger dans l'espace, c'était une chose. Disposer d'une réserve d'énergie suffisante pour ne pas transpercer sa planète de destination à une vitesse -relativement- proche de celle de la lumière, c'en était une autre. Les glorieux scientifiques qui avaient résolu ce problème en doublant la taille des vaisseaux afin de pouvoir accumuler suffisamment d'énergie pendant le voyage s'étaient trouvés trop heureux de leur élégante trouvaille pour prendre ensuite en compte des éléments futiles tels que le confort auditif des passagers lors d'une telle débauche de puissance.


Les nouveaux et anciens habitants de la planète aux Iles n'entendaient rien de ce grondement mais ils levèrent tous des yeux émerveillés vers le ciel lorsque le vaisseau entra dans l'atmosphère. La nature particulière des gaz émis par le vaisseau les faisaient réagir sous l'impact de l'énergie thermique dégagée en ionisant les particules atmosphériques. Un Terrien n'aurait pas pu évoquer autre chose qu'une aurore boréale accélérée et aussi colorée qu'un arc-en-ciel pour décrire le phénomène qui emplit alors le ciel. Mais les longues traînées d'une couleur unie, nées d'une réaction en chaîne se propageant entre les gaz de même nature, qui semblaient couler avec la vivacité d'une cascade tout en s'entortillant autour d'obstacles invisibles et mouvants - des poches de gaz neutres à la réaction en cours - échappaient à toute comparaison.


La douce lueur bleutée du soleil qui éclairait la planète aux Iles depuis que l'union de deux galaxies l'avait projetée dans son champ de gravitation, quelques millions d'années auparavant, n'était plus visible qu'autour du vaisseau humain, protégé par un champ d'argon. La majorité des créatures pensantes de la planète, concentrées sur les îles les plus massives, ne bougeait plus. Les yeux écarquillés vers le ciel, elles observaient l'immense vaisseau noir sombrer lentement vers elles, entouré de rayons bleus pâles dont l'éclat disparaissait rapidement, emporté dans le tourbillon de couleurs qu'était devenu le ciel. La dernière phase de l'atterrissage impliquait la libération des restes d'énergie stagnant dans les accumulateurs de l'appareil. Et tous purent voir, alors que les tourbillons colorés se fondaient les uns dans les autres, alors que d'immenses langues éclatantes se distendaient vers le sol jusqu'à pendre à quelques dizaines de mètres de celui-ci, d'immenses arcs électriques traverser fugitivement le ciel depuis le vaisseau et enflammer les dernières poches de gaz restées inertes. Les filaments enflèrent autour de ces longues bulles de feu et furent bientôt absorbés par la combustion.


Quelques secondes plus tard, tout était terminé. Le ciel était à nouveau pur, bleu pâle, et percé d'une forme sombre aux contours anguleux. De nombreuses personnes clignèrent des yeux avec la sensation de sortir d'un rêve. Les habitants et les visiteurs se regardèrent, et la grande trahison reprit sa marche implacable.


Isaac était malade. Une immunodéficience pulsative étalée sur des périodes de plusieurs années rythmait sa vie depuis qu'il était né. Comme de nombreux autres individus de sa génération, il n'était que le résultat raté d'expériences menées sur des embryons à une époque où de nombreux parents avaient cru à la toute-puissance des nouvelles méthodes de manipulation génétique in utero. Le génome d'Isaac était globalement instable, mais les seules conséquences observables survenues jusqu'à présent étaient ces périodes durant lesquelles son système immunitaire s'effaçait totalement. Isaac savait qu'un jour, les produits chimiques et les radiations modulées dont on l'inondait ne suffiraient plus à enrayer l'autodestruction programmée de son organisme. Mais il s'était construit une vie, à la fois par instinct mais aussi soutenu par l'espoir secret que les traitements, qui accentuaient l'instabilité de son génome, finiraient un jour par engendrer une mutation qui résoudrait ses problèmes.


Couché sur une civière coiffée d'un demi-cylindre transparent, Isaac sentit son corps s'incliner et vécu à nouveau sa chute du haut de sa couchette. Il sursauta, extirpé de sa torpeur, et une vague de douleur déferla en lui laissant derrière elle une sensation de brûlure dans sa poitrine et les épaules. La civière glissait sur la rampe du vaisseau. Encadrée de lourdes machines, elle était retenue dans sa chute par un infirmier squelettique. Les veines de ses bras saillaient sous l'effort, et un Iliian se précipita pour l'aider lorsque l'équipement médical du vaisseau eut touché le sol humide de la planète aux Iles. Ils poussèrent la civière contenant le corps tuméfié sur les herbes épaisses qui jonchaient le sol en une quantité telle que la marche en devenait difficile.


Un véhicule de transport militaire surgit en marche arrière d'une large brèche dans le mur en béton qui encerclait la zone préparée pour l'atterrissage du vaisseau. L'infirmier monta avec la civière, et l'Iliian salua avec déférence les militaires sortis du camion pour soulever Isaac et son cocon de plastique.


Derrière eux la cérémonie, réduite à sa plus simple expression, put enfin commencer. Des hommes en costume élégant, une écharpe bleue en travers de la poitrine, s'avancèrent sur la rampe à pas lents, leur visage composé, le regard porté vers l'horizon.


- Ils en font des tonnes, souffla le commandant Avon au capitaine Fetcher.

- Je pense que j'en ferais autant en découvrant la planète que je m'apprête à gouverner, mon commandant, lui répondit Fetcher. Il y avait un soupçon d'envie dans sa voix.

- Administrer, capitaine. Nous ne sommes pas chez nous, voyons. Du coin de l'œil, Avon vit un sourire ironique s'étirer sur les lèvres du capitaine. Il ne put s'empêcher de sourire à son tour.


Le vaisseau contenait, outre les membres du gouvernement chargés d'administrer les futures zones civiles, de nombreux experts venus régler tous les détails de construction et d'interaction avec les autochtones - ainsi que quelques scientifiques qui avaient réussis à obtenir une longueur d'avance sur leur collègues, s'offrant ainsi la primeur des informations intéressantes qu'ils pourraient exploiter dans leurs travaux au sujet de ce nouvel écosystème. Je me demande si ils savent à quel point on se fout de leur avis, songea le commandant Avon. Tous ces détails ridicules avaient soit déjà été réglés par ses hommes, soit ne méritaient pas que l'on s'y attarde.

***

La raison principale de la venue des experts était tout simplement de tester le taux d'adaptabilité d'un civil à ce lieu. Il arrivait qu'une planète, aussi adaptée à l'organisme humain qu'elle soit, ne soit en pratique pas vivable. Les psychologues prenaient alors l'affaire en main et expliquaient plus ou moins bien pourquoi les civils n'avaient pas tenus plus de deux mois à sa surface. Il arrivait que la réponse soit évidente, comme un paysage profondément glauque et dérangeant selon les normes culturelles terriennes ; mais ce n'était pas toujours le cas.


A ce sujet, l'avis personnel du commandant Avon était que la plupart des gens n'étaient rien d'autre que des lavettes sans volonté, à peine capables d'apprécier tout ce que le gouvernement leur offrait sur un plateau. Mais le commandant était très au fait de la nature polémique de ses avis personnels, et c'est pourquoi il se gardait bien de les énoncer - tout en les appliquant à la lettre chaque fois qu'il en avait la possibilité.


Lorsque les dignitaires eurent fini de parader, tous les regards se tournèrent vers Avon. L'autorité officielle des nouveaux arrivants n'était rien comparée à la puissance de celui qui connaissait et maîtrisait parfaitement le terrain. Avon était, pour quelques semaines encore, le maître incontesté de cette planète. Mais lorsque le contrôle militaire s'effriterait peu à peu, laissant place à une organisation composée de notables, d'experts et d'immigrants, il disparaîtrait dans l'oubli. Conscient de cette lutte de pouvoir embryonnaire, le commandant choisit soigneusement la teneur de son discours.


- Monsieur le gouverneur, Messieurs les ministres. Je vous souhaite la bienvenue, au nom de mes hommes et moi-même, sur la plus grande découverte de l'humanité depuis le moteur à explosion. Vous savez déjà à quel point ce lieu représente une soupape de sécurité pour toute notre civilisation. Mais cette planète n'est pas seulement composée à quatre-vingt pour cent de ressources naturelles indispensable au fonctionnement de nos vaisseaux et de nos véhicules. Elle n'est pas seulement l'endroit où nous viendrons nous approvisionner lors d'une éventuelle pénurie. Vous avez déjà compris, je l'espère, les mécanismes de son écosystème si particulier. Les composés carboniques enfouis sous nos pieds ont atteint une masse critique qui accélère la décomposition des organismes environnants. Pour faire simple, nous sommes énergétiquement sauvés.


Mais cette planète est aussi incroyablement vivable. Respirez. Observez la pureté de l'air. Je suis sûr que même avant l'arrivée de l'homme, la Terre n'était pas aussi pure. Bien sûr, l'air va progressivement se charger de particules et nous devrons un jour revêtir les mêmes combinaisons ultra résistantes que celles que nous portons pour nous déplacer à la surface de notre bonne vieille Terre. Mais nous avons le temps ! Nous avons malheureusement perdu le compte des années, mais notre planète bleue a été exploitée durant au moins trois cent ans. Et aujourd'hui, en 360 après LA, voici une deuxième Terre qui s'offre à nous. Imaginez, Messieurs ! Imaginez nos descendants, dans de nombreuses générations ! La Terre a été vidée. Elle se nettoie lentement. Dans quelques milliers d'années, elle redeviendra l'Eden qu'elle était avant... eh bien, avant nous. Nous n'avons qu'à exploiter des ressources telles que celles sur lesquelles vous vous tenez en ce moment, et toutes nos planètes, une fois exploitées et nettoyées, seront autant de havres de paix où les hommes pourront vivre. Un avenir radieux s'offre à nous.


Mais il me semble que vous jetez des regards intrigués vers nos amis ! Laissez-moi vous présenter les Iliians. Ces créatures, tout comme nous, sont imprégnées de tolérance, de respect pour son prochain, et ne souhaitent qu'une chose : participer à l'amitié entre nos deux espèces. Comme le Traité sur la prise en considération des autochtones non-humains doués d'intelligence l'indique, il est de notre devoir, sauf conditions exceptionnelles, de veiller à ce que nos relations soient basées sur des échanges réciproques, à une transparence totale dans nos opérations sur leur territoire, et à la non-exploitation de leurs ressources naturelles sans leur consentement éclairé.


Ne vous inquiétez pas, les Iliians sont peut-être la meilleure chose qui pouvait nous arriver. L'exploitation de leurs sols ne les dérange pas dans certaines proportions et en échange d'un approvisionnement continu en éléments de la faune et la flore humaine. Vous le savez déjà, puisque ce vaisseau est rempli à ras-bords de plantes et d'animaux de toutes les espèces encore vivantes sur Terre. Ce marché avantageux, nous le devons au degré d'évolution relativement inférieur des Iliians. Ils n'ont pas,  à notre connaissance, de grands artistes ni de philosophes. Ils passent leur temps à nous observer et nous aider, et je dois avouer que leur déférence surclasse parfois celle de mes assistants ! Mais l'expansion de leur espèce ne les intéresse pas. La colonisation n'est pas une de leurs valeurs morales. En fait, tout ce que ces créatures veulent, c'est en savoir le plus possible sur l'univers et le monde environnant. L'exploitation de ce savoir n'a aucun intérêt pour eux !


Cette planète est notre idéal. Messieurs, ne foirez pas tout. Observez. Posez-moi des questions. Si je suis ici pour quelques semaines encore avant de vous passer la main, c'est pour une bonne raison. Et n'oubliez pas le traité de prise en considération. Il serait dommage de devoir quitter cette planète parce que nous avons dépassé les quotas d'exploitation des sols fixés par nos charmants hôtes.


Le message était passé : Ce coin était un nid à intrigues politico-économiques et la meilleure chose à faire était d'éviter à tout prix d'avoir à essuyer les plâtres ; ce qui signifiait laisser le contrôle au commandant aussi longtemps qu'il le souhaiterait. Avon perçu dans les yeux des dignitaires un mélange d'admiration et de haine et en tira un immense plaisir. Il venait de créer le mélange parfait, grâce auquel il pourrait continuer sa mission sans interférences.

\s

Une immense tache blanche disparaissait peu à peu. Derrière elle, un patchwork flou apparu lentement, au fur et à mesure que la douleur diminuait dans le crâne d'Isaac. Un bloc vert se déplaçait à la périphérie de son regard. Derrière lui, des bandes oranges et grises se disputaient son espace visuel avec une immobilité remarquable.


Isaac redécouvrit l'usage de son cristallin. Le bloc vert rétrécit de plus en plus vite, et des plis apparurent à sa surface. Un individu en blouse verte se penchait au-dessus d'Isaac.


- Professeur ? Vous m'entendez ?


Isaac contracta son larynx et se frotta à un tube en plastique. Il en ressentit une gêne extrême mais savoura l'absence de douleur. Il leva les yeux au ciel, dépité.


- On vous a intubé, professeur. Serrez ma main si vous m'entendez. Parfait.


L'homme en blouse recula et saisit un boîtier de commande relié à un gros câble qui pendait du plafond. Autour de la lampe qui l'avait ébloui lors de son réveil, trois pointeurs à l'extrémité bleutée s'éteignirent dans un sifflement. Puis la pièce fut plongée dans le noir. Isaac reconnu le bruit d'une perfusion que l'on change et s'endormit soudain.


\s

- Où en est notre philosophe ?

Fetcher leva les yeux de son carnet de notes et regarda son commandant avec une indifférence composée.

- Etat stable. Les radiations fonctionnent de moins en moins bien si j'en crois son carnet de santé. Il tiendra une semaine avant d'avoir besoin d'une autre séance.

Posant deux mains larges et poilues sur son bureau d'acier, le commandant se leva avec lenteur, comme si la gravité avait subitement triplé.

- Très bien ! Soupira-t-il. Je vais rendre une petite visite à ce pauvre homme et voir ce qu'il souhaite faire avant de nous quitter.

Fetcher tiqua et se replongea dans ses notes. La mort était un sujet qui l'embarrassait au plus haut point. Surtout lorsqu'il devait se confronter à elle par le biais de quelqu'un sur le point de passer l'arme à gauche. Avon observa son capitaine et haussa les sourcils. Cet homme n'a décidément pas ce qu'il faut pour monter en grade avant une bonne dizaine d'années. Il attrapa sa veste dans un geste élégant, la jeta sur ses épaules et sortit d'un pas vif.


Le bureau de travail du commandant et de son capitaine contenait à lui seul la définition du mot fonctionnel. Deux tables métalliques criblées de tiroirs trônaient au fond de la pièce, disposées en angles droits. Des chaises pliantes étaient réparties du côté de la porte et un grand tableau blanc occupait le mur de gauche. Celui de droite était recouvert de cartes et de plans détaillant les installations militaires en place et les coordonnées des différentes stations d'extraction disposées sur l'île principale. Entre les chaises inconfortables et la porte vitrée, trois armatures sur roulettes soutenaient un nombre impressionnant de casiers. Le commandant Avon évitait autant que possible les systèmes informatiques. Ses assistants étaient là pour s'occuper de ce genre de choses.


Le seul élément superflu du bureau était le fauteuil du commandant. Fetcher se leva à son tour et marcha lentement vers celui-ci. Arrivé devant l'écrin de velours, il s'immobilisa et tendit l'oreille. Personne ne se déplaçait dans le couloir adjacent. Le capitaine s'installa confortablement et s'enfonça doucement dans le fauteuil. Un immense sentiment de satisfaction pénétra en lui. Les énormes accoudoirs auraient facilement pu accueillir deux personnes. Fetcher les parcourut des doigts, les caressa pensivement. Puis il ferma les yeux et savoura cet instant. Quelques secondes plus tard, il était de nouveau sur sa chaise, concentré et efficace.


\s


Avon entra en trombe dans la chambre d'Isaac. Adressant simultanément son sourire le plus chaleureux au malade et un signe clair en direction de la porte à l'infirmier de garde remplaçait la perfusion d'Isaac, il s'avança vers le lit et prit place sur le tabouret le plus proche.

- Monsieur Blau ! Ravi de vous rencontrer. Comment vous sentez-vous aujourd'hui ?

Isaac fronça les sourcils.

- Qui êtes-vous ?

Avon sursauta. Les médecins devaient avoir oublié quelques tumeurs au fond de son crâne.

- Cher monsieur, je suis le commandant en chef de cette base, et par extension de cette planète. Vous pouvez m'appeler Stephen.

Isaac Blau battit lentement des paupières et se détourna vers la fenêtre. Placée au dernier étage de l'hôpital militaire, sa chambre lui permettait d'embrasser le paysage monotone et relaxant de la planète aux Iles. Au-delà du camp, agrégat de bâtiments administratifs en préfabriqué et de tentes luxueuses, s'étendait à perte de vue un immense champ d'herbes épaisses. D'un vert si foncé que les prairies les plus lointaines se confondaient avec les eaux sombres des lacs avoisinant la côte, ces fougères locales constituaient la végétation dominante au sein de l'étrange écosystème découvert par les hommes lors de leur arrivée.

- Monsieur Blau ?

- Combien de temps ? Isaac observait toujours l'extérieur, le visage inexpressif. Seuls ses yeux éteints, peu attentifs au décor, reflétaient la résignation qui grandissait en lui.

- Un mois. Peut-être deux. Vous êtes déchargé de toutes vos obligations et pourrez passer le temps à venir comme bon vous semblera.

Isaac se tourna brusquement vers le commandant. Ses mâchoires étaient contractées à l'extrême, ses paupières plissées. Son corps entier semblait s'être tendu en l'espace d'une fraction de seconde.

- Il n'en est pas question, commandant. J'ai construit ma vie sur un socle en cristal. Ce n'est pas parce que je le vois se fendiller que je vais m'arrêter pour autant.

Le commandant Avon soutint le regard d'Isaac.

- Très bien. Je vous réquisitionne. Vous allez finir votre existence en découvrant les arcanes de la colonisation. De l’intérieur.

-Votre offre ne m’intéresse pas.

-  Monsieur Blau ! Ne me faites pas croire qu’un gamin né au milieu de la folie déclenchée par la découverte d’autres espèces intelligentes n’est pas devenu philosophe parce que ça le passionnait.

- Je n’ai pas de comptes à vous rendre. Je suis venu ici pour assister les zoologistes, et c’est ce que je ferais du temps qui me reste.

- Ce n’est pas comme si vous aviez le choix, mais j’espère que vous savez que vos dernières semaines seront les plus ennuyeuses de votre triste existence. Aucun livre de zoologie ne sera jamais publié ici. Un mémo, à la rigueur. Ou un post-It. En cumulant la faune et la flore, on arrive au nombre impressionnant de deux espèces. Même sur Terre, il en reste plus.

- Comme si j'avais le choix ?

- Oh, excusez-moi. Etant donné votre état, je ne devrais pas être étonné. Laissez-moi vous rafraîchir la mémoire. Je suis le patron. Vous êtes mes fidèles serviteurs. En-dessous, il y a les Iliians, mais ça, c’est un autre problème. Lorsque vous pourrez à nouveau marcher, votre première destination a intérêt à être mon bureau.

Isaac se laissa doucement tomber sur son oreiller. Il ferma les yeux et attendit un moment. Lorsqu’il les rouvrit, le commandant était toujours là. Les bras croisés, assit sur sa chaise, il regardait froidement Isaac.

- Pourquoi me forcer ainsi ?

- Parce que c’est ce que vous voulez. Je me fiche que vous l’admettiez. Mais je ne vous laisserai pas gâcher votre dernière chance de réaliser un rêve pour de stupides questions de principe.

- Qui ont construit mon existence !

Avon se leva.

- Elle touche à sa fin, votre existence, dit-il. Puis il s’en fut.


Quelques minutes plus tard, Isaac s’assit et tâta ses jambes. Pas d’éruptions cutanées. Il descendit jusqu’à ses genoux, tâta le creux poplité. Aucune douleur. Il essaya de bouger ses orteils et soupira en les entendant craquer.

Il avait bien quelques heures avant qu’un infirmier ne se rende compte qu’il allait mieux. Il allongea ses jambes confortablement et profita de ces derniers instants de repos.


\s


Fetcher attendait son commandant derrière la porte. Celui-ci hocha la tête et dépassa son capitaine. Lorsque leurs visages furent au plus près, Avon murmura :

- Il est du voyage.

Fetcher sourit ostensiblement et lui emboîta le pas. Il sentit un énorme poids s’envoler de ses épaules pour aller s’abattre discrètement sur celles du philosophe condamné.

\s

Isaac Blau observait les herbes. Rien n’était plus monotone et reposant que ce paysage naïf. Debout sur le seuil de l’hôpital, il détaillait les grappes de feuilles qui jonchaient le sol. En un sens, ces plantes ressemblent à des algues, songea-t-il. Les grosses feuilles molles poussaient en cercle autour d’un fin réseau de racines blanchâtres enchevêtrées. Isaac plia les jambes et appuya sur l’un d’eux du bout de l’index. Un liquide transparent et huileux suinta d’entre les fibres.


-  Surprenant, n’est-ce pas ?


Un homme était apparu subitement au côté d’Isaac. Ce dernier leva la tête et découvrit un individu grand et mince, les lèvres serrées dans un rictus que même un enfant n’aurait pu confondre avec un sourire. Ses mains jointes devant lui, crispées à l’extrême, composaient un damier rouge et blanc tant elles étaient serrés.


Isaac se leva, gardant les yeux fixés sur les mains de l’individu. Il les vit se séparer et rougir subitement sous l’effet du sang qui se remettait à circuler dans les doigts. Isaac leva enfin son regard vers l’homme. Son visage était un paquet de muscles contractés.


- Vous m’avez l’air nerveux, dit Isaac.

-  Surprenant, n’est-ce pas ? répondit l'inconnu.


Isaac se demanda un instant si des yeux pouvaient jaillir de leurs orbites. Ceux de l’inconnu devaient être en train de battre un record.

- Vous…vous allez bien monsieur ?

- Nerveux ? L’inconnu parcourut plusieurs fois du regard un public invisible. Ah !

L’homme fouilla frénétiquement dans sa poche et en extirpa une boîte de pilules. Ce n’est qu’à ce moment qu’Isaac remarqua que l’homme portait un galon sur son épaule. Mais il ne connaissait rien à la signification des grades.

Le militaire avala trois pilules et poussa un soupir.


- L’huile. Vous avez déjà vu une plante pareille sur Terre ou dans les livres d’Histoire ?

-  J’espère bien que non. Six mois dans un cercueil géant, ça vaut bien un peu de nouveauté à l’arrivée.

- Mais c’est qu’on garde le sens de l’humour !

Isaac ne pu réfréner un élan d’agressivité.

-  Qui êtes-vous ? Comment savez-vous ?

 - Capitaine Fetcher. Enchanté.

- C’est difficile ?

- De quoi ?

- De parler à un mourant.

- Comment…comment savez-vous que je sais ?

Isaac savoura le renversement des rôles.

   -Les pilules que vous venez de prendre ont agi immédiatement, donc vous n’en prenez pas souvent. Et juste après, vous vous permettez de plaisanter sur ma santé. A moins que vous ne soyez un salaud fini, ça veut dire que mon triste état vous trotte dans la tête. Perché sur un marteau-piqueur.

Fetcher ouvrit la bouche et cligna plusieurs fois des yeux. Aucun son ne se décida à quitter sa bouche.

- Que disiez-vous à propos de l’huile ?

-  Il…transforme toutes nos plantes en ces espèces d'algues vertes.

- Quoi ?

- Les premiers tests ont eu lieu ce matin. Un simple contact cutané transforme le végétal en moins de deux heures.

-  Vous voulez dire que…

- Non. Nous sommes ici depuis un an et je peux vous assurer que ces plantes n'ont jamais eu le moindre effet sur nous. J’ai pensé que ça pourrait vous intéresser.



Fetcher salua Isaac et tourna les talons. Alors qu’il avait atteint le pas de la porte de l’hôpital, Isaac l’interpella. Fetcher pila mais ne se retourna pas. Ils savaient tous les deux que le capitaine était venu parler à Isaac pour une autre raison. On ne court pas chercher des philosophes pour leur annoncer le résultat d’expériences médicales. Fetcher poussa la lourde porte de l’entrée principale et disparut.


Isaac plissa les yeux à son tour et secoua la tête avec dépit. Il s’assit sur la marche la plus basse du parvis et arracha une feuille du sol. La pétrissant entre ses mains, il s’interrogea longuement sur ce que le capitaine était venu lui dire. A moins qu’il n’ait seulement désiré voir le philosophe malade.


Mais pourquoi quelqu’un d’aussi mal à l’aise avec la mort se forcerait-il à rendre visite à un mourant ?


\s

L’Iliian ne bougeait plus. A cinq centimètres de son visage, un tube de métal tremblait. A l’autre bout du tube, le commandant Avon suait. Les yeux de l’Iliian s’agrandirent encore un peu. Avec la forme de son visage, la lueur dans son regard et sa tête légèrement penchée sur le côté, il battait à plate couture une portée entière de bébés chats. Le mot mignon avait rencontré sa première définition en chair et en os.


Stephen Avon se surprit à pencher la tête à son tour. Avec une moue dégoutée, il se dressa de toute sa hauteur et envoya le tube de fer claquer contre le mur le plus proche. Le bruit fit sursauter l’Iliian et une infime touche de tristesse vint se fondre parmi ses traits. Sous la fureur, les narines d’Avon s’agrandirent. Il serra les poings et les ramena contre son front. Fermant les yeux un instant, il les rouvrit avec une fureur décuplée, comme si le simple mouvement de ses paupières conférait à ses yeux à nouveau dévoilés une force nouvelle et brutale. Le sang lui battait aux tempes si puissamment que les bruits environnants résonnaient au rythme de son cœur, l’enfermant encore plus dans sa propre rage.


Fetcher entra dans la pièce. Lorsqu’il croisa le regard d’Avon, il fut si terrifié par la haine qu'il y trouva qu’il ne remarqua pas la nuance de mépris glacé que s’était mis à exprimer le commandant à sa vue. L’Iliian s’était lentement tourné vers Fetcher dans une supplication silencieuse à laquelle nulle créature douée d’émotion n’aurait dû pouvoir résister.


-  Dehors, dit-il dans un souffle.


L’Iliian abaissa ses frêles épaules, fixa le sol durant un court instant et articula avec désolation :


-  Oui.

Fetcher attendit qu’il ait quitté la pièce et se dirigea vers Avon, qui s'était assis dans son siège régalien et contemplait une carte des régions environnantes.


- Que se passe-t-il ?

-  Ils l’ont fait exprès. Ils ont modifié leurs plantes.

-  Eux ? Pour quoi faire ?

- Ils nous surveillent. Ils nous contrôlent.

- Les Iliians sont incapables d’une chose pareille. Vous le savez aussi bien que moi.

- Je les hais. Encore plus que toutes les autres espèces, les autres étranges créatures que j’ai pu rencontrer jusqu’à aujourd’hui. Cette vermine est si attentionnée, si sensible… je n’ai même pas pu le frapper.

-  Le… vous êtes fou ?

Le commandant tourna brusquement la tête vers Fetcher, le transperçant du regard.


- Je veux dire, commandant, vous connaissez les conséquences qu’un tel incident pourrait engendrer.

 -  Ce n’est définitivement pas le problème. Ces choses sont trop…douces, trop attendrissantes. Ca me met hors de moi.
- Nous n’aurons bientôt plus…

Le bruit feutré d’une porte glissant sur une moquette épaisse interrompit brutalement Fetcher. Il jeta un coup d’œil derrière son épaule et aperçut le vieux philosophe s’avancer péniblement vers eux, avec un air ingénu qui contrastait vivement avec son état de santé.


-  Blau !

- Isaac me plaisait mieux.

-  Bien sûr, Isaac. Veuillez m’excuser.

 Fetcher laissa le silence se prolonger quelques secondes avant de prendre la parole.


- Nous avons beaucoup à faire, commandant. Les généticiens ont demandé une réunion pour cet après-midi et les centres d’extraction veulent que vous passiez les briefer individuellement sur les nouveaux quotas.

- Dans quelques minutes. J’ai une question à poser à monsieur Blau. Que pensez-vous des Iliians ?

- Je ne suis pas encore entré en communication avec l’un d’eux. Mais j’imagine qu’ils feraient de très bons jouets pour enfants.

- Vous les appréciez ?

- Je ne crois pas qu’un Iliian se permettrait de décréter s’il aime ou non l’humanité après nous avoir côtoyé durant quelques jours, commandant.

Subitement, Isaac s’étrangla et porta la main à sa poitrine. Tombant à genoux, il toussa et cracha un sang noir et épais. Levant la tête un instant, il vit les deux hommes, restés immobiles à l’autre bout de la pièce. Ses yeux s’agrandirent avec frayeur.

- Je crois que vous devriez retourner à l’infirmerie, dit Avon d'un ton glacé.


Isaac hocha plusieurs fois la tête et se précipita hors de la pièce.


- Vous me ferez nettoyer ça, continua Avon à l'intention de son capitaine.

- Très bien, commandant.

- Et veillez à ce que monsieur Blau prenne son traitement aux heures prévues.

 -  Comme si sa vie en dépendait, commandant.


\s


Dans les couloirs du quartier général, Isaac se sentit rapidement mieux. Une sensation de flottement le poursuivait, mais après des années à suivre des traitements lourds, il avait appris à s’accommoder de ce sens de l’équilibre approximatif et des pensées engourdies qui accompagnaient les antidouleurs. Concentré sur son épaule qui frôlait l’un des murs, savourant la sensation de sécurité que lui procurait cet appui fiable, Isaac failli manquer l’Iliian qui s’approchait face à lui. L’étrange curiosité du commandant avec piqué la sienne en retour, et il sentait qu’une conversation avec l’un des autochtones l’aiderait à comprendre ce qui se nouait sur cette planète.


  -  Vous !


L’Iliian s’arrêta et pencha doucement sa tête sur le coté. La simplicité physique de l’être faisait écho à la plénitude du sentiment de sympathie que sa personne exprimait. Isaac laissa son regard glisser sur la créature vêtue d’une sorte de poncho blanc cerné de multiples ceintures. Il ne parvenait pas à déterminer ce qui créait cette curiosité amicale, ne pouvant se résoudre au fait que de découvrir enfin quelque chose de simple puisse le combler à ce point.

  - Avez-vous un nom ?

Avait-il posé cette question ? Quelle était la réponse, d’ailleurs ?

Soudain, il ressentit une pression sur son flanc gauche. Une tumeur rapide venait de se former. Le souvenir de sa condition réveilla immédiatement en lui des images de chambres stériles et de scalpels. Une chaîne de souvenirs, depuis sa douleur jusqu’à son identité, se déroula dans son esprit et délia enfin sa langue.

   - Je suis…

   - L’enfant malade.

   - Qu’est-ce que l’enfant malade ?

    -  Le dernier-né.

    -  De quoi parlez-vous ?

L’Iliian bascula vivement sa tête de l’autre coté et esquissa un sourire chaleureux.

     - Celui qui naît toujours vivra. Celui qui agite en lui le passé des siens restera. L’enfant malade reste seul.


Isaac tâta son flanc gauche. La tumeur avait disparue. Avait-elle jamais été là ? Il souleva sa blouse et observa une formation calleuse entre son bassin et ses côtes. La parcourant des doigts, il reconnu le relief d’écailles se chevauchant.


Il voulut parler à l’Iliian de nouveau, mais celui-ci avait disparu.


Revenu dans sa chambre, Isaac gratta un peu de son dérèglement cutané et le plaça dans une capsule à côté de ses échantillons sanguins prêts à être analysés. Mais la litanie fanatique de l'Iliian l'avait intrigué, et il passa la moitié de la nuit à bouger en tous sens, incapable de trouver le repos.

\s

Les installations pétrolières brisaient le paysage. La tranquillité d’Isaac s’évapora à mesure qu’il s’approchait des armatures de métal scintillantes et l’imminence de sa fin surgit dans son esprit comme une réalité inacceptable sur laquelle le climat de la planète aux Iles avait jeté un voile à présent en lambeaux.


Dans le véhicule, Isaac faisait face au commandant. Celui-ci souriait béatement et ses narines dilatées inspiraient l’air vicié avec l’empressement de celles d’un homme qui vient de s’échapper de l’eau in extremis. Derrière eux, deux traînées sombres s’effaçaient lentement, le liquide extrait des plantes lors du passage du véhicule tout-terrain s’enfonçant à nouveau dans le sol.


- Ces végétaux sont de vraies éponges à hydrocarbures. Le jour où on ne pourra pas creuser plus profond, on n’aura qu’à pomper directement à cette source. Leurs racines s’enfoncent entre les couches du manteau et renouvellent le cœur de cette foutue planète. Mais il y a déjà d'énormes poches éparpillées sous la surface, donc on n’utilisera pas cette méthode avant d’avoir épuisé les moyens classiques. Trop cher pour l’instant.


Isaac n’était pas vraiment intéressé. Circulant entre les centrales, il vit de nombreux Iliians travailler, transportant du matériel vers les chantiers de constructions. Il se demanda à nouveau d’où venait l’étrangeté de ces êtres. Même en ce moment de trouble, le cœur crispé autour de l’orbe noir de sa mort, l’empêchant de grandir, de se répandre en lui, de l’engloutir avant l’heure, un doux réconfort le caressait alors qu’il observait leur paisible simplicité. Leur apparente union lui donna le sentiment d’être dans l’erreur la plus profonde, de ne pas pouvoir saisir l’insignifiance de sa disparition face à la survie de son essence, contenue dans chacun de ses frères humains.


Le véhicule s’arrêta et tous descendirent. Lorsqu’Isaac posa le pied au sol, un Iliian apparu à ses côtés et lui soutint le bras.

-  Observe et souviens-toi, enfant malade.

-  Comment savez-vous que je suis malade ? Commandant !


Mais le commandant et le chauffeur étaient loin devant, et ne semblaient pas avoir remarqué qu’Isaac ne les accompagnait pas. Celui-ci scruta le visage de l’Iliian avec nervosité. Il ne rencontrait qu’un calme confiant, une quiétude, à la poursuite de laquelle l’humanité courait depuis qu’elle avait contemplé ses propres mains avec étonnement pour la première fois.


-  Comment savez-vous ?

-  Tu es l’enfant condamné. Tel tu resteras.

-  Non, comment savez-vous que je suis malade !

-  Vous l’êtes tous. Mais tu es le seul nouveau-né.

-  Je suis un homme condamné à mourir, pas un enfant.


L’Iliian pencha sa tête sur le coté. Isaac reconnu clairement la condescendance de l’aîné face au cadet impulsif, enfoncé dans l’erreur.


-  L’homme en toi devra disparaître pour que l’enfant s’éveille.


Isaac souleva sa chemise et présenta son flanc gauche à la créature.


-  Et ça, qu’est-ce que c’est ?


L’Iliian tendit sa main menue, couverte d’une légère fourrure, dont les contours rappelaient celle d’une patte d’ours en peluche. Il effleura les écailles, et sourit.


-  C’est l’homme qui rebrousse chemin.


Et il s’en fut.


\s


Les installations étaient titanesques. Les bâtiments extérieurs n’étaient que la partie visible d’un complexe souterrain où les véhicules étaient construits, les hydrocarbures raffinés et conditionnés en unités d’énergie. Le commandant présenta le lieu à Isaac avec autant de fierté que s’il l’avait construit de ses propres mains, exprimant le génie humain ici à l’œuvre comme une évidence qu’il n’était pas la peine de mettre en avant. Mais les quotas semblaient mettre Avon hors de lui. Il s’excita, criant pour couvrir le bruit environnant.

-  Regardez ces machines ! La ligne rouge, c’est le débit maximum autorisé par ces foutues créatures. Plus de quatre-vingt pour cent de notre capacité dort encore.

-  Ca n’accélérerait pas la destruction de la planète ?

-  Vous plaisantez ? Vous voyez quelque chose de plus important que l’expansion de l’humanité ? Enfin, c’est notre raison d’être ! Que ce soit par la bonne parole, la technologie ou notre façon de vivre, c’est pour ça que nous sommes programmés mon cher, nous répandre !


Isaac leva les yeux et imagina, au-delà du plafond, de la roche, des êtres en surface, de l’espace, la Terre telle qu’il pourrait en recevoir l’image. Des créatures humanoïdes cherchant à survivre au sein d’un cycle immense, seules accablés par le poids supplémentaire de la conscience de soi.


La visite touchait à sa fin. Le commandant lui posa la main sur l’épaule et, avec l’air de celui qui racontera tout plus tard, lui glissa :


-  J’ai quelques ordres à donner, je vous rejoins tout à l’heure. Je vous laisse avec Cole, c’est une jeune apprentie.

 -  Qui ça ?


Le commandant pointa une jeune femme du doigt parmi un groupe d’individus très affairés à discuter autour d’un café. Elle répondit par un hochement de tête désabusé et peu protocolaire.

Le commandant jeta un coup d’œil à Isaac et croisa son regard. Il toussota, et lâcha avant de partir :


-  Attendez une minute, elle va arriver.


Isaac sourit avec bienveillance en regardant l’homme s’éloigner.


\s


Jane Cole finit de briser délicatement le sucre, le mélangea au liquide et tendit la main vers un pichet de lait en métal. Au passage, elle arrêta la machine à café et enleva la capsule écrasée du réservoir. Puis elle versa le lait dans la première tasse, se tourna gracieusement vers la table derrière elle et, soufflant avec un air concentré sur le thé à la menthe qu’elle venait de préparer, alla jusqu’à Isaac et lui posa la tasse sous le nez. Puis elle bondit vers le comptoir, saisit son expresso et revint s’asseoir prestement, un sourire chaleureux subitement apparu sur son visage large.


Isaac dut presque se réveiller tant le rythme de ses pensées avait ralenti dans l’observation de la jeune femme. Il saisit sa tasse avec précaution et, levant les yeux pendant qu’il sirotait son thé, il la vit porter le regard en haut à droite et faire la moue. Une demi-heure de conversation lui avait suffit pour repérer ce tic qui survenait à chaque fois qu’une idée lui traversait l’esprit et qu’elle examinait celle-ci brièvement avant de déterminer que ce pourrait être un sujet de réflexion intéressant à partager. Il anticipa le plaisir d’entendre sa voix rauque et chaleureuse à nouveau.


-  Vous savez ce qui m’étonne le plus avec ces Iliians ?

-  Leur gentillesse ?

-   Oh non, la gentillesse, c'est bizarre pour nous, mais je ne vois pas pourquoi ça ne serait pas la norme pour une autre culture. C’est surtout leur simplicité qui me surprend.


Isaac interrompit sa gorgée pour gratter son flanc droit où le relief écailleux s’était étendu un peu plus. Il songea que comparé à son génome instable, les humains étaient eux-mêmes des animaux plutôt simples.


-   Vous ne trouvez pas ? dit-elle en avançant son visage, un sourire amusé aux lèvres. Visiblement, elle croyait qu’il avait perdu le fil de la conversation.

-   Je ne sais pas, de quelle simplicité parlez-vous ?

-  Ca va vous sembler bête, mais il va falloir que je vous parle de mon enfance pour que vous compreniez !


Elle rejeta sa tête en arrière et poussa un soupir.


-   Je ne sais pas non plus en fait. Ca me fait toujours penser à lorsque j’étais malade. Quand vous êtes petit, vous restez installée dans votre lit et plus rien ne repose sur vous. Enfin, il y a peu de choses qui reposent sur les enfants, mais ça devient comme absolu…vous êtes relégué au rang de spectateur et supposé apprécier cette position. Vous restez un humain, un membre de la famille, mais la vie extérieure ne vous concerne plus. On vous en protège, parce que vous êtes malade.


Incapable de voir un lien avec la possible simplicité des Iliians, Isaac concentra toute son attention sur la jeune fille, ignorant les sourdes douleurs qui s’éveillaient un peu partout dans son organisme.


-   Enfin bref, vous êtes là, bien installée, et la seule chose qu’on vous demande c’est de survivre. On veut toujours sauver les malades parce que si ils survivent, c’est toujours comme si tout le monde avait survécu, surmonté l’épreuve, et prouvé que l’homme est quand même vachement fort. Et lorsque j’étais petite, je rentrais dans ce rôle et je passais mon temps à observer. Et à chaque fois je réalisais quelque chose que j’oubliais dès que j’étais guérie. Et les Iliians me l’ont rappelé.


Elle se leva et se prépara un deuxième café. Le dos tourné, elle interrompit sa discussion et ne put voir Isaac soulever sa chemise et inspecter ses côtes. Elle n’entendit pas non plus le hoquet de stupeur et de dégoût qu’il fit lorsqu’il remarqua qu'un lichen vert avait fait irruption entre ses côtes. Il appuya dessus, et un liquide sombre suinta.


Cole se retourna une seconde trop tard et ne vit que le teint livide du philosophe. Mettant cela sur le compte de son état de santé, elle continua son histoire.


-  Donc voilà, il y a cette réflexion que je me faisais petite : les gens sont compliqués. C’est ça qui m’a frappée chez les Iliians, ils sont simples, dépouillés, alors que nous, on est pleins de…d’inutile. Et j’étais là, à observer les actions des gens d’une façon différente, j’avais l’impression de voir le vide derrière leur mouvement, de ne plus faire partie du tourbillon. Et alors que je les jugeais, eux me transformaient en une sorte de réceptacle, comme si ma survie était le garant de la leur, la preuve du bien fondé de leur existence. Et tout ça était enrobé d’agitation vide dont je faisais partie à nouveau dès que je recouvrais la santé.


Isaac avait une longue expérience de la douleur. Mais cette fois était radicalement différente. Tout en essayant de garder une contenance face à la charmante ingénieur, il avait désespérément tenté de faire abstraction de ce qu’il avait vu émerger à la surface de sa peau. Mais ce qui irradiait maintenant à travers sa chair n’avait rien de la désagréable sensation communément étiquetée « douleur » et possédant divers degrés d’intensité. Les images d’une pique de glace et d’une langue de feu se confondirent lorsqu’il tenta de visualiser ce que ses nerfs lui transmettaient. Mais ce malaise insupportable qui se propageait lui donnait aussi l'impression de perdre la propriété de lui-même. Il existait un sentiment de propriété et d’appartenance, une reconnaissance de lui-même lorsqu’il pensait à chaque partie de son corps. Et ce qu’il expérimentait était la destruction de cette identité. Il pouvait presque visualiser l’intérieur de son corps lui devenir étranger, continuer de fonctionner en symbiose avec son cerveau par de simples relations chimiques et lui laisser un semblant de vie, mais il ne reconnaissait plus sa personne physique comme cette partie de son identité qui accompagnait son esprit depuis plus de cinquante ans.


Il se leva, chancelant. La table et Coler tournoyèrent dans un vertige halluciné et Isaac ne la vit pas s’approcher à toute vitesse pour le rattraper.


\s


A son réveil, Isaac était à nouveau dans sa chambre. Ces épisodes d’inconscience commençaient à l’agacer.


Fetcher entra en trombe.


-  Votre état ne s’arrange pas, dit-il, et Isaac le méprisa pour oser proférer de pareilles énormités.


 - Vous êtes plein à ras-bord de produits divers et variés pour le moment. Mais votre prochaine séance de rayonnements doit avoir lieu dans les deux heures qui viennent. Pourriez-vous vous rendre à la dernière salle au bout du couloir d’ici deux heures ?


- Très bien. Laissez-moi me reposer à présent.


Isaac ne se reposa pas. Les propos de la jeune fille, qu’il avait ignorés sur le moment, obnubilé par sa douleur, lui apparurent teintés de mystère, inexplicablement lié à certains aspects de sa situation

\s

Retraçant ce qu'il avait vécu depuis son arrivée, il ne put formuler une structure, un plan qui réunirait les événements dont il avait été témoin et leur donnerait une cohérence. Mais du fond de son hébétude, alors que le cheminement de sa pensée s'arrêtait et reprenait en un mouvement erratique, du fond de son incompréhension et de la peur sourde qui naissait de cet état auquel il n'était pas accoutumé, Isaac perçut deux forces immenses allant lentement à la rencontre l'un de l'autre. Il s'imaginait les sentir, tournoyer autour de lui, créant un œil de paix au cœur de ce cyclone intangible, traversé de courants irrésistibles, faits de volonté et de haine, de certitude et d'anxiété, d'ambition et d'arrogance. Il se vit choisir, pressé de désigner. Le choix, pourtant, lui sembla truqué. Hors de tout langage, vide de toute parole, il vit deux chemins s'ouvrir et l'appeler.


A l'un d'eux il se préparait depuis toujours. Il s'agissait d'un fleuve noir à la surface ridée par des vagues de néant. Chacune d'entre elles exhalant un flot de terreur avant de disparaître, nourrissant une éternité de non-être.


L'autre voie était une chute vertigineuse au fond de laquelle son identité s'écraserait. Mais la mort était absente. Isaac se pencha au-dessus de l'à-pic et se vit tomber. Emportées par la vitesse, des parties de son corps s'envolaient, dévoilant une carcasse couverte d'yeux, d'écailles, de plumes, de fourrures, d'os, d'arête, de membranes, de dents et d'antennes. Et lorsqu'il heurta enfin le sol, il entendit son humanité périr.


Isaac laissa la douleur devenir insupportable. Il voulait voir ce que deviendrait son corps, libéré des stabilisants qu'on lui avait injectés. Il contracta ses abdominaux et vit le relief de nouvelles écailles se dessiner sous sa peau. Son maigre thorax était à présent lisse et huileux. L'un de ses tétons avait disparu. Tourmenté par le feu que ses nerfs supportaient, il se leva enfin et se dirigea vers la salle où Fetcher lui avait donné rendez-vous.

\s


Ouvrant péniblement la porte, il ne remarqua pas le corps étendu au sol. Il se dirigea vers le lit, et ses pieds nus trempèrent dans une flaque de sang. Il sauta en arrière et ses muscles hurlèrent leur souffrance. Sans même baisser les yeux, il avait reconnu la consistance poisseuse du sang fraîchement répandu. Au sol gisait Fetcher, les mains serrées autour de sa gorge inondée d'hémoglobine. Ses yeux exprimaient la surprise et le dégoût avec tant de force qu'Isaac douta pour un instant qu'il fût bien mort. Il reprit son souffle et, tremblant, se dirigea vers l'émetteur de rayonnements. Le choc avait atteint son cœur et il pouvait à présent le sentir se défaire, se transformer en un amas de cellules chaotiques. Incapable de s'expliquer comment il pouvait encore se mouvoir, il enclencha la machine et s'effondra sous ses rayons bienfaisants.


Lorsqu'il se réveilla, l'odeur du sang embaumait la pièce jusqu'à l'écœurement.


Isaac éteint les machines et se tourna vers le corps. À sa place se trouvait une fine pellicule de sang séché recouverte de rectangles numérotés. Il leva les yeux vers la porte et aperçut le dos de deux officiers. Alors qu'il se dirigeait vers eux, il vit le commandant s'avancer depuis l'extrémité du couloir sur lequel donnait la pièce. Les gardes s'écartèrent, et le commandant entra.


- Que s'est-il passé ?


Un silence flotta. Ils avaient parlé en même temps. Isaac reprit.


- Je me rendais ici pour une séance de radiations quand j'ai trouvé le capitaine. J'ai voulu vous avertir, mais mon état était trop grave et il me fallait cette séance.

- C'est ce que nous nous sommes dit en vous découvrant. Votre faiblesse vous a innocenté dès le départ, et ça a pas mal accéléré l'enquête.

- Vous avez trouvé le coupable ?

- Suivez-moi, répondit Avon.


Il sortit vivement de la pièce. Suffisamment pour qu'Isaac se demande s’il voulait réellement lui montrer quelque chose ou si il était seulement incommodé par l'odeur.


Ils marchèrent dans le dédale de la base militaire. Les autres hommes, invisibles jusque-là, étaient devenus omniprésents. Deux militaires gardaient chaque porte. Les scientifiques étaient tous accompagnés. Aucun Iliian n'était visible. Une légère démangeaison poussa Isaac à porter une main au sommet son crâne. Ses doigts effleurèrent une crête osseuse. Il digéra le changement avec calme et se concentra sur ce qui semblait être leur destination.


Le commandant saisit la poignée de la porte blindée, approcha la main d'un scanner digital, et s'arrêta un instant. Sans se retourner, il murmura: "Enfin, nous y sommes".

Une fraction de seconde avant que la porte ne s'ouvrît, Isaac sut ce qu'il allait découvrir. Et toute une partie de lui, devenant par la même occasion plus forte, écrasant un peu plus ce qu'il savait être quelque chose que bientôt il ne se souviendrait plus avoir été, trembla et gronda, rugit et cria, et déversa dans ses veines une haine sereine, une aversion absolue pour ce qu'il était en train de cesser d'être, et que seraient toujours les colonisateurs de cette planète. Il cilla alors que le parfum de trahison répandu sur ce monde se révélait enfin à lui, et vit ce à quoi le commandant l'avait mené.


Attachés aux murs de l'immense pièce dans laquelle Isaac venait de pénétrer, des Iliians le regardaient fixement. Leurs corps étaient ouverts, brisés ou couverts d'ecchymoses. Certains saignaient encore abondamment ; d'autre, n'était leur tête tournée vers Isaac et leur regard à la simplicité transperçante, arboraient toutes les caractéristiques du cadavre.


Isaac demanda au commandant, dont le visage reflétait sa stupéfaction face à la réaction collective qu'il avait observé lorsqu'Isaac était entré dans la pièce :

- Combien de temps a-t-il fallu à la machine pour me soigner cette fois ?

- Trent...trente-six heures.


Les Iliians contemplèrent Isaac un moment de plus, puis leurs muscles se relâchèrent brusquement et ils s'affaissèrent, désarticulés. Avon et Isaac échangèrent un regard ambigu, à la fois de connivence dans l'incompréhension où ils se trouvaient face à ce mouvement inexplicable, mais aussi teinté de suspicion envers Isaac pour Avon, et d'une noire rancœur, envers Avon, pour Isaac. Le commandant s'avança entre les autochtones torturés et déclara à Isaac, maintenant quelques pas derrière lui, qui marchait plus lentement et examinait les corps avec amertume :


- Les Iliians nous ont trahis. C’était assez évident, à bien y réfléchir. Un peuple n'offrirait jamais ses ressources tel qu'ils l'ont fait pour nous sans avoir un dessein caché.

- Quel dessein ?

- Obtenir nos espèces pour satisfaire leur curiosité inutile et perverse je présume. Toujours est-il que Fetcher a vu ce qu'ils faisaient aux animaux. Nous avons récupéré plusieurs spécimens complètement transformés. Impossible de savoir ce qu'ils étaient avant, mais ils ressemblaient tous plus ou moins à un quadrupède à la peau translucide avec de grands yeux noirs d'Iliian. Quand aux plantes, elles s'étaient toutes transformées comme par magie en cette espèce d'algue verte qui pousse à la surface et d'où ce dérivé du pétrole suinte à chaque fois qu'on appuie dessus.

- En quoi cela prouve-t-il leur duplicité ?

- Vous croyez que je vais faire confiance à des créatures qui font subir des traitements aussi ignobles à d'autres êtres vivants ? Cette espèce est pourrie de l'intérieur, mon cher Isaac.



Isaac frissonna en entendant "mon cher" sortir de la bouche du commandant à son intention.


- Oh, et vous n'allez donc sûrement pas utiliser cette méthode pour générer du combustible ?


Blau contracta ses lèvres juste assez pour avoir l'air à la fois dégouté et méprisant.


- Vous n'êtes pas foutu de faire la différence entre une putain de curiosité perverse et des besoins énergétiques élémentaires ? Evidemment que ce phénomène sera utilisé de façon profitable. Enlevez vos œillères cinq minutes, bon sang.


\s

Isaac n'avait pas revu les politiciens depuis son arrivée. Les scientifiques étaient trop heureux de découvrir ce nouveau bac à sable biologique pour s'embarrasser de scrupules. Les militaires semblaient avoir pleinement embrassé la nature graisseuse, servile et éteinte de leur cerveau. Avon était l'empereur de ce nouveau monde.


Ce dernier tira Isaac par le bras et le dirigea vers la sortie. Ils marchèrent jusqu'à son bureau en silence. Isaac observait ses pieds, aussi muet intérieurement qu'extérieurement. Il jetait parfois un regard en direction d'Avon. Une jubilation difficilement retenue déchirait le visage du commandant. Lorsqu'ils entrèrent, Avon marcha à grandes enjambées jusqu'à son fauteuil, s'y affala et, souriant, poussa un soupir de contentement. Grattant nerveusement son torse, Isaac attendit qu'il s'ouvre à lui.


- Cher ami ! Voici enfin ce que je vous avais promis lors de notre première rencontre. Assoyez-vous, je vais vous expliquer l'opération unique que j'ai mise en place dans le seul but de garantir la suprématie de notre civilisation.


Isaac pencha sa tête sur le coté et ne dit mot. Il vit une gêne passagère glisser sur le visage du commandant et cligna plusieurs fois des yeux. Lentement, il avança jusqu'au siège de Fetcher. Il posa une main sur l'accoudoir, regarda le commandant à nouveau, et dit :

- Je vais rester debout.

Il alla jusqu'au fond de la pièce, effleura le mur, puis fit un demi-tour et s'appuya sur le dos, bras croisés sur la poitrine. Avon, qui avait fait pivoter son fauteuil pour suivre le déplacement du philosophe, entrelaça ses doigts velus et déposa son menton sur l'arche ainsi formée.


- Les Iliians n'ont rien à voir avec la mort de Fetcher.


\s

Dans les stations pétrolières, l'activité avait triplé. Des corps d'Iliians étaient empilés dans des cuves, attendant d'être couvertis en énergie pour soutenir le nouveau rythme de production. Les équipes de manutention avaient la nette impression d'être tombés dans un puits de travail au fond improbable ; depuis qu'ils avaient dû ajouter à leurs tâches celles que les autochtones avaient délaissées. Dans chaque unité, l'aiguille du cadran général se déplaçait. En 12 heures, quatre-vingt dix pour cent des capacités de production étaient entrées en fonction, et moins de trois jours seraient nécessaires pour atteindre une productivité complète. Des patrouilles lourdement armées sillonnaient les scintillations, à la recherche d'hypothétiques Iliians échappés. En vain. Lorsque l'opération de sécurisation avait commencée, les ingénieurs avaient trouvé tous les natifs déjà alignés devant les cuves de conversion énergétiques, paisibles, exprimant plus que jamais cette troublante sérénité qui caractérisait les Iliians. 


Les communications avaient été coupées avec les Iles périphériques, et la dernière étape du plan de sécurisation, différée de trois jours, consistait en un bombardement systématique du reste de la planète, qui abritait la majeure partie de la population Iliian.


\s


Avon riait. Cramponnées à son bureau, ses deux mains vibraient au même rythme que ses joues flasques et son ventre gras. Face à lui se tenait une maigre créature au corps et à l'esprit torturés. Son immobilité semblait être en lutte directe avec l'excitation destructrice du commandant. Sous ses vêtements, sa peau était devenue un champ de bataille organique. Les démangeaisons dont il souffrait provenaient plus d'un dégout pour cet état de transition abject aux yeux de l'humain qu'il était que d'une quelconque cause physique. Sous sa peau, ses organes tremblaient sous l'impact d'un chaos aux origines intangibles. Au plus profond d'Isaac, sa nature, autant innée que patiemment construite, tombait en lambeaux pour devenir une matrice au destin étranger à elle-même.


- Pourquoi Fetcher ? demanda Isaac d'un ton monocorde.


L'amas de haine joviale que formaient les traits du commandant s'éclaira un peu plus.

- Pour régler deux problèmes en une fois ! Cette tantouze psychorigide voulait mon poste à tout prix. Je suis à fond pour la sélection naturelle, mais cette fiotte n'aurait jamais fait le travail correctement. Et il aurait fait n'importe quoi pour me remplacer. Comme il était au courant de mon projet pour cette planète et qu'il savait que je ne le portais pas dans mon cœur, je l'ai rassuré en vous mettant dans la partie. Pour lui, c'était vous la pauvre victime qui allait se faire tuer par les Iliians.


La raison pour laquelle Fetcher était venu le voir dans un tel état de nervosité sembla soudain limpide à Isaac. Ca doit être difficile de rencontrer l'homme qu'on a désigné pour mourir à sa place, songea-t-il.


\s


Isaac évoluait dans son environnement avec la circonspection et la curiosité coupable qu'il aurait éprouvé en visitant une cathédrale érigée à la gloire de la démence. Passionné par l'inhumanité d'Avon, il suivait ses voyages d'apparat chez les scientifiques, les politiciens et ses fidèles soldats. Les Iliians étaient dépecés les uns après les autres, incinérés, dissous, maintenus conscients durant des séances de radiations mutatives, dégradés en composés organiques minimums, puis exécutés et convertis en énergie après avoir vu leur génome numérisé. Isaac, du fond de son château de souffrance, observait le mouvement humain avec surprise. Il s'interrogeait sur la superfluidité de leurs manières, la vanité de leurs désirs, la finalité de la pente sur laquelle tous glissaient, roulaient puis grossissaient.


Les données s'accumulaient au sujet de l'organisme Iliian. Isaac s'asseyait aux tables rondes des chercheurs et, les yeux dans le vague, écoutait les découvertes les plus récentes. L'organisme Iliian était composé d'un nombre restreint de gènes. Le nombre de segments non-codants était d'un ordre de grandeur insignifiant, si bien que chaque individu était l'expression de la majeure partie du génome qu'il possédait. Les spécimens étaient aussi très peu différenciés les uns des autres. Au-delà des mutations accidentelles, aucun gène possédant des variations dominantes et récessives n'avait été observé. La société Iliian semblait avoir été aussi uniforme socialement que biologiquement.


Chaque particule de pétrole était stockée avec soin dans une grille de stockage auto-génératrice. Les unités de quarante mètres cube chacune étaient formées de huit générateurs de pression chacun capable de retenir un quart de chacune des trois faces à l'angle desquelles ils étaient disposés. Une fois pleines, les unités pouvaient être déformées et inclinées par une commande à distance qui influait sur la répartition de la pression extérieure. Les unités pleines servaient alors à faciliter le transport du matériau brut dans des récipients classique et chaque jour, la grille se remplissait plus vite que la veille. Observant le ballet de cubes noirs qui venaient prendre position au sein de la grille avec un air ému, Avon soufflait : "Attendez de voir le vaisseau cargo embarquer tout ça !"


Les Iliians manquaient à Isaac. Il n'était pas torturé par la culpabilité, conscient du fait que révéler le complot n'aurait que précipité sa mise sous calmants pour mourants en plein délire. Mais leur présence lui manquait tout simplement. Il lui semblait être le dernier à poser un regard interrogateur sur la diversité, la violence, l'arrogance, la destruction humaine; et pourtant il pouvait voir en lui-même cette agitation caractéristique qui le troublait bien plus que sa déchéance physique ; et c'était l'absence d'une telle agitation qui avait procuré, lorsqu'il était au contact des Iliians, un sentiment de plénitude à Isaac.

\s

Un soir, Isaac emprunta à Avon l'une des bibles qu'il lisait régulièrement. Pour l'avoir étudié par le passé, il parcourut l'ancien testament sans le lire réellement, se perdant plutôt en considérations sur le sens de son contenu, ses yeux glissant sur les versets et sa mémoire faisant ressurgir les images qu'il avait formées des années auparavant.


Alors qu'il lisait l'histoire du déluge, Isaac se demanda qui était le véritable responsable du cataclysme. L'idée que Noé était l'agent décisif qui avait permit la mort du reste de l'humanité se libéra en lui comme l'inattendu parfum d'une nouvelle fleur, à la fois trop étranger à soi pour être immédiatement adopté mais trop tentant pour ne pas mériter une étude plus poussée.


Peu à peu, il se prit à voir Noé comme celui qui, en acceptant de construire son arche, avait abandonné sa propre humanité. S'élevant loin de ses semblables il avait, en garantissant la poursuite d'un nouvel idéal humain, laissé leur mort survenir. Alors qu'Isaac considérait les conséquences qu'auraient eu un refus de la part de Noé de construire son arche, laissant à Dieu le choix de détruire définitivement le but de son existence ou de le laisser dégénérer, il reçut un appel d'Avon.


Isaac voulut décrocher, mais la communication fut interrompue brutalement. Il se leva douloureusement et marcha à pas lents jusqu'au bureau d'Avon. Refoulant avec lassitude les pensées morbides qui ne cessaient de surgir en lui à mesure que sa fin approchait, il pressa fermement sa main sur la vitre sans teint qui séparait le monde extérieur du centre de pouvoir dont Avon était le locataire.


La porte n'était qu'entrouverte lorsqu'il put discerner le commandant, debout à cinquante centimètres de lui. Tout son visage bougeait de façon désordonnée, comme s'il avait tenté d'articuler des paroles avec l'ensemble de ses muscles faciaux mais s'était retrouvé victime d'une crise de bégaiement. Ses yeux écarquillés et presque larmoyants semblaient bien décidés à fuir leurs orbites pour se précipiter à l'extérieur du bureau. Isaac n'éprouva qu'une tranquille curiosité. Il pencha sa tête sur le coté et s'enquit :


- Vous m'avez appelé, commandant ?


Lentement, par à coups presque mécaniques, les lèvres d'Avon s'écartèrent et il parvint à répondre :


- Non... c'est lui qui voulait que vous...


Il interrompit sa phrase et indiqua du regard quelque chose à sa gauche. Isaac passa sa tête par l'ouverture et se tordit le cou. A l'autre bout de la pièce se trouvait un corps écartelé, dont l'identité ne pouvait être déterminée plus précisément que par le fait que ce corps avait appartenu à un membre de l'espèce humaine. Le visage était fendu en deux dans le sens de la hauteur, le cerveau s'étendant encore, déchiré, au travers de l'espace qui les séparait. Les habits du cadavre, étalé sur le ventre, étaient couverts de tellement de sang qu'il était peu probable qu'il en restât encore à l'intérieur, et une ouverture béante dans son dos laissait entrevoir une cage thoracique brisée. Les organes internes, autrefois spécialisés par leur forme et leur fonction, n'étaient plus qu'un amas de composés organiques suintant hors du corps, traçant des sillons clairs dans la mare d'hémoglobine.


Isaac contempla cette œuvre un instant et retourna à Avon.


- Vous allez devenir barge si vous continuez à tuer vos camarades pour justifier la mort de vos ennemis, command...


Un éperon venait de surgir entre les yeux d'Avon. Il descendit d'un mouvement uniforme jusqu'à son entrejambe, remonta d'un coup sec et finit de diviser son corps en éclatant le sommet de son crâne par une dernière accélération. Alors que les deux moitiés tombaient lentement au sol, leur contenu commençant déjà à se déverser entre elles, deux pattes arachnéennes les saisirent en un geste transversal dont la dextérité n'échappa pas à Isaac. Finissant leur trajectoire circulaire, elles laissèrent les restes du commandant leur échapper et ceux-ci achevèrent leur course contre la baie vitrée du bureau, grossièrement réassemblés en un simulacre d'humanoïde à l’étrangeté repoussante, telle une figure de cire que l'on aurait laissé trop longtemps au soleil. Derrière leur passage s'était dessiné un rideau ocre, suspendu dans les airs pour une fraction de seconde. Lorsqu'il s'affaissa, en un temps qui à Isaac parut une éternité, le prédateur se révéla enfin.


La créature possédait un corps gris marbré de noir. Les zébrures palpitaient au travers de sa peau suivant une pulsation irrégulière plus proche du rythme d'un drapeau claquant au vent que de celui d'un cœur aux battements réguliers. Une excroissance vaguement sphérique distendait sa cage thoracique dont la surface semblait avoir été transpercée par des filaments blancs d'épaisseur variable. Certains pendaient, apparemment inutiles, tandis que d'autres liaient la sphère à plusieurs des membres du monstre. Nombre d'entre eux décrivaient un arc avant de revenir s'insérer au niveau de la zone correspondant aux épaules et au cou humains. Une faible quantité s'était entorsadée autour du corps du monstre et, d'un diamètre plus important que la moyenne, se trouvait à moitié enfoncées dans sa chair.


Le monstre avait la posture d'un gorille mais semblait capable d'adopter une palette de mouvements bien plus complexe que ce dernier. En fait de bras, chaque côté de son corps possédait deux appendices qui se rejoignaient en une seule griffe à la consistance proche de la pierre, mais recouvrant sept membres articulés disposés en demi-cercle dont la nature digitale ne faisaient aucun doute. Encore plus menaçant que le reste, son crâne large et bas possédait un sommet convexe et une mâchoire inférieure qui rappela à Isaac les mythiques poissons des profondeurs dont la Terre avait réalisé l'extinction bien avant qu'elle ne puisse en tirer plus que quelques photographies saisissantes.


L'être détendit son bras et projeta ce faisant une traînée ocre et visqueuse qui constella le sol et la baie vitrée du bureau. Isaac s'approcha mécaniquement de celle-ci pour pouvoir observer ce que le monstre désignait en contrebas. Lorsque le sol entra dans son champ de vision, ses muscles se tétanisèrent. Les vitres parfaitement insonorisées l'isolaient du carnage qui se déroulait sous ses yeux, mais l'imagination du philosophe recréa le bruit des balles et de la chair broyée. Une horde de créatures semblables à celle qui avait tué le général s'affairait à éliminer toute vie humaine de la base, dans un ballet de violence et de mort contre lequel les armes automatiques des soldats ne pouvaient offrir qu'une résistance aussi symbolique que dérisoire.


Isaac se détacha de cette vision, à la fois effrayé du spectacle et des sentiments inattendus qui éclosaient en lui. Le mot harmonie lui apparut, à sa grande horreur, et écœuré par sa réaction inhumaine face au massacre dont il venait d'être le témoin, il resta stupéfié, mentalement immobilisé par ce conflit intérieur qui se manifestait une fois de plus, lui laissant un sentiment d'inquiétante étrangeté d'autant plus terrible qu'il se sentait basculer du côté le moins humain du dilemme. Soudain, il remarqua que l'être qu'il fixait sans le voir depuis maintenant quelques secondes se transformait. Un à un, les filaments les plus fins se détachèrent et, tels des serpents de ficelle palpitante, se tordirent et retournèrent au centre de sa poitrine. Ceux plus larges suivirent l'empreinte formée dans la chair du monstre, qui se referma après leur passage. Quelques secondes plus tard, le corps de l'être commença à changer. Ses griffes mollirent, puis rétrécirent jusqu'à n'être plus qu'une légère bosse sur le plat de sa main à six doigts. Les bras et les jambes rétrécirent jusqu'à rappeler un humanoïde chétif, et alors que sa mâchoire se fondait dans sa cage thoracique pour en dévoiler une seconde plus petite et dépourvue de dents, la créature pencha sa tête sur le côté et planta son regard dans celui d'Isaac avec une expression de compassion infinie.


Le philosophe ne fut pas réellement surpris de l'identité entre les Iliians et les monstres qui avaient surgit du néant, mais il ne put s'empêcher de trembler alors qu'il soutenait le regard de la créature la plus dangereuse et la plus adorable qu'il ait jamais rencontré.


Un dernier filament pendait hors de la poitrine de l'Iliian. Il fit un pas vers Isaac, et le filament se redressa pour venir toucher le cou de ce dernier. Puis il se rompit, et serpenta dans le vide jusqu'à ce qu'il ait totalement pénétré le cou d'Isaac. Celui-ci resta interdit, mais aucune sensation n'accompagna cet échange inquiétant. Brutalement, Isaac ressentit un mouvement surgissant de l'intérieur de son corps qui se propagea jusqu'à son épiderme. Un second mouvement, telle une vague nerveuse, se répandit depuis son coccyx jusqu'à la base de son cou. Réalisant que sa peau se déplaçait, Isaac arracha sa chemise tout en s'écartant vivement de l'Iliian qui n'avait pas bougé d'un millimètre et observait Isaac avec curiosité. Dehors, le carnage avait cessé et le camp n'était plus qu'un charnier informe. Les mains d'Isaac s'étaient palmées. Il les porta au niveau de ses yeux et observa le nouvel aspect translucide qu'avait revêti sa peau. Son dos le démangeait. Il se tordit afin de pouvoir se gratter à son aise et ramena une poignée de plumes. Il pouvait à présent identifier la sensation comme étant celle d'un réseau de cartilage surmonté de plumes s'extirpant doucement de son dos. Les derniers instants virent le cartilage rompre sous son propre poids et tomber lourdement au sol, laissant le dos d'Isaac à vif. Baissant à nouveau les yeux vers ses mains, il aperçut son torse et ses jambes, dont la surface était à présent proche de la peau d'un batracien. Peu à peu, des écailles se dessinèrent à leur surface, et telles un bas-relief devenant de plus en plus profond, elles possédèrent bientôt tout le contraste et le détail de véritables écailles de poisson. Isaac en saisit une au niveau de son abdomen et tira. L'écaille suivit, créant un pont fait d'une substance gélatineuse entre la peau d'Isaac et elle. Prit d'une crise de frayeur, il les frotta de façon hystérique, et c'est avec l'impression que tous ses ongles lui étaient arrachés simultanément qu'il les fit tomber au sol. Revivant l'histoire de l'écosystème où son espèce avait évolué, Isaac sentit une queue émerger du bas de son dos, des griffes grossières surgir de ses doigts qui lui arrachèrent un cri d'horreur et de douleur, une crête déformer son crâne, l'un de ses doigts tomber et repousser, l'une de ses dents s'allonger jusqu'à entailler son menton, ses organes internes se déplacer en tous sens, les ongles de ses pieds percer ses chaussures d'hôpital et, devenu noirs, peser si lourd que ses jambes déjà presque inutilisables se brisèrent sous ses efforts pour se déplacer de quelques centimètres. Lorsqu'il tomba, ses nouveaux appendices se détachèrent et furent accompagnés par son nez, ses oreilles et ses cheveux. Sa vision obscurcie par la douleur lui permit néanmoins de voir l'Iliian se pencher au-dessus de lui et il l'entendit murmurer :

- "Tant de déchets, enfant malade"

\s

Lorsque l'Iliian se réveilla, il sut tout de suite qu'il ouvrait ses yeux sans iris pour la première fois. Il se redressa au milieu des lambeaux de chair humaine qui lui avaient servi de paillasse durant sa nuit de repos et observa alentours. Un semblable se tenait à la limite de l'ouverture rectangulaire séparant deux zones de confinement artificielles créées par des visiteurs à présent disparus. Au fond de son corps, l'Iliian ressentait les restes d'une sauvagerie arrogante, chargée des fardeaux du passé de ses ancêtres. Il inspira profondément et savoura la simplicité de son nouvel esprit. Elargissant ses perceptions à son environnement immédiat, il se découvrit n'être qu'un insignifiant composant de la beauté de celui-ci. Les créatures chaotiques venues visiter son espèce avaient été détruites avant qu'elles ne puissent étendre l'incendie stérile qui s'allumait dans leur sillage. Passé le temps d'observation, il s’était avéré que la branche maîtresse de leur écosystème était impropre à une vie harmonieuse avec le reste de la réalité. Par chance, cette branche possédait un sujet apte à l'épuration. Celui-ci avait été tué, mais par une méthode assurant la conservation de ses mémoires et schémas mentaux afin de ne pas perdre définitivement tout reliquat de la présence des humains dans l'histoire du monde. La Grande Trahison avait finalement eu lieu.

Il fallait à présent effacer le danger humain. Depuis des Iles mineures avaient décollé plusieurs centaines de vaisseaux lourdement chargés de spores toxiques en direction de la Terre. Une fois cette planète sauvée, le tour des colonies humaines viendrait, et l'harmonie serait enfin retrouvée.


Cessant de penser à une échelle collective où son identité n'était qu'un éphémère concept soumis aux fluctuations de l'espèce, sa matrice, l'Iliian découvrit qu'il était toujours entaché de ce qui avait autrefois été Isaac Blau.


Isaac Blau n'avait jamais cessé d'être une entité fluctuante, soumise aux caprices de sa nature synthétique formulée avec insouciance par d'apprentis démiurges. Il se réveilla à nouveau et constata que son corps lui était plus encore étranger qu'il ne l'avait été par le passé. Il avait en outre le souvenir d'avoir été autre, plus fourni en émotions et contradictions. Le monde lui semblait à présent limpide et calme.


Il rejoint le flot de pensée Iliian et les signes et images qui y flottaient le rappelèrent douloureusement au massacre qui avait été perpétré sur ceux qui avaient jadis été ses frères. Il comprit alors que sa situation était inattendue. Iliian de corps et d'esprit, son génome réduit à un chef-d’œuvre de minimalisme débarrassé des contingences de l'évolution, le cancer humain survivait en son sein et contrôlait les centres définissant son identité.


Le lieu où se trouvait Isaac avait perdu de l'importance. Chaque individu vivait simultanément à travers les yeux du reste de l'espèce, une expérience commune qui faisait de l'isolation physique des Iliians une caractéristique auxiliaire, mince accident structurel dénué d'importance face à l'existence collective perçue par les habitants de la planète aux Iles.


Isaac ne pouvait toutefois détacher entièrement le lien qui existait entre son enveloppe corporelle et l'idée d'un soi autonome et solitaire. Les restes humains malencontreusement oubliés au cœur de ses fonctions cérébrales n'y étaient pas innocents et la stupeur latente provoquée par l'étrangeté de ce nouveau corps et du mode de pensée inhabituel qui l'accompagnait ne faisaient qu'affaiblir la force qui unissait Isaac à ses frères Iliians.


Mû par une injonction intérieure, Isaac marcha jusqu'à la salle où le général et lui avaient observé les expérimentations menées sur des Iliians quelques jours auparavant. Les corps n'avaient pas été déplacés et Isaac put se complaire une fois de plus dans l'observation curieuse de l'imagination humaine libérée de tout carcan moral. Toujours fixés aux murs par des cercles de métal, les cadavres avaient commencé à se décomposer en un mucus grisâtre qui ne suscita chez le nouvel Iliian aucun sentiment nauséeux, le mucus ressemblant tant à un matériau susceptible de devenir la matrice d'autres organismes qu'il n'évoquait en rien la perte et le dégoût que colportent souvent les restes humains.


S'arrachant à ce spectacle, Isaac s'assit au centre de la salle et accéda aux consciences de ses nouveaux frères. La facilité avec laquelle il s'immergea dans cet océan d'esprits lui fit perdre pied ; et Isaac mourut alors, devenant tout à fait Iliian l'espace de quelques secondes. Puis il ressurgit avec force, surnageant au-dessus des autres, possesseur du pouvoir d'interagir avec l'espèce Iliian sans y dissoudre son identité.


S'imprégnant des nuées flottantes autour de lui, Isaac découvrit qu'une multitude de présences se trouvaient masquées au plus profond de la planète. S'enfonçant dans le bourbier organique, il arriva à la source responsable de l'état de la planète aux Iles. A l'inverse des éléments spatiaux connus des hommes, la planète aux Iles possédait un noyau de nature organique. Un cœur vivant, composé d'êtres unicellulaires et d'immenses racines perçant la terre jusqu'à éclore sous forme de plantes huileuses à sa surface battait là, se nourrissant des cadavres que les Iliians se chargeaient d'acheminer, les faisant descendre le long de galeries situées au pôle opposé à celui où les humains avaient établi leur camp de base. Les résidus végétaux remontaient le long des racines, emplissant le sol de restes dont l'état de décomposition le plus avancé, proche de la surface, avait attiré les humains et créé de grands espoirs pour le futur énergétique de leur espèce.


Isaac contempla ce système avec une admiration teintée de tristesse devant l'impression de vacuité que dégageait cet équilibre parfait, organisme complexe en harmonie avec le reste de l'univers. Le chemin parcouru par ses composants était fait d'une tranquillité si glorieuse qu'il était devenu son propre but, cercle dont le sens était restreint à lui-même et au vide révélé par sa perfection. Isaac pouvait aisément séparer ses sentiments humains de sa nature Iliian, observer la façon dont ils parasitaient sa sérénité et le poussaient à sortir, s'échapper, courir vers un perpétuel ailleurs dont la seule promesse était celle d'un nouvel horizon vers lequel s'élancer, à nouveau et toujours. Isaac eut alors le sentiment que dans cet affrontement entre ses facettes Iliians et humaines, création et destruction s'unissaient du côté humain pour combattre l'harmonie éternelle des Iliians et forcer l'apparition du neuf au cœur de l'ancien.


La nature des plantes huileuses était à présent claire aux yeux d'Isaac. Pendant végétal du philosophe, elles étaient originaires d'un organisme instable réduit à sa plus simple expression par l'influence Iliian. Au contact d'autres végétaux, l'instabilité se transmettait, détruisant la nature de la cible, la transformant de la même façon que l'Iliian qui avait tué le général avait tenté de faire d'Isaac l'un des siens en jouant de son inconstance génétique pour lui faire perdre ses attributs inessentiels. Au fond de son être, Isaac ressentait un semblable pouvoir. Il se leva brusquement et marcha jusqu'à la sortie du camp humain.


La possibilité qu'il puisse soustraire les humains à leur sort en détruisant leur essence plutôt que leur corps l'obnubilait. Il accéda brièvement à la conscience commune et repéra l'emplacement des navettes de transport rapide. S'il se dépêchait, elle arriverait avant la flotte Iliian.


Après s'être procuré la grenade la plus puissante qu'il puisse trouver, il suffit à Isaac de faire quelques pas hors du camp pour découvrir en lui-même les nombreuses routes secrètes mises en place par les Iliians après l’arrivée des humains. S'échapper de l'environnement où il était mort renforça son être nouveau-né, et tout le savoir Iliian ressurgit comme une pensée perdue sur laquelle on a longtemps cherché à mettre la main avant de renoncer. Ce qu'il se représenta comme un hangar d'appoint était enterré près d'un lac dans lequel il dut s'immerger afin d'accéder au boyau de roche dissimulé derrière des algues et quelques amas rocheux. Alors qu'il dégageait l'entrée, puis, plus tard, se faufilait à la nage jusqu'au hangar, Isaac songea que les Iliians faisaient bien peu de cas de la sécurité de leurs installations et, interrogeant la part autochtone de son être, comprit que leur curiosité seule avait offert un sursis aux humains.


Le conduit débouchait sur une poche exiguë vraisemblablement peu propice à la réception de vaisseaux interstellaires. Isaac se hissa hors de l'eau et inspecta la grotte ruisselante dans laquelle il était arrivé. Déposant sa grenade à terre, il se dirigea vers le fond du cul-de-sac, où se trouvait une installation aux contours peu discernables. Laissant à ses yeux le temps de s'habituer à l'obscurité quasi totale, il s'approcha du fond de la grotte et inspecta sa découverte. Apparemment faite de tissus organiques, l'installation sortait du sol telle une excroissance violacée et avait la forme d'un large bassin aux bords épais, surmonté d'un arc de chair au sommet duquel un filament blanc ondulait, poussé par une brise qui expliquait la présence d'oxygène dans ce réduit sous marin.


Reconnaissant l'un des appendices qui avaient permit aux Iliians de devenir des monstres invincibles, Isaac réalisa que son caractère humain avait interprété les informations qu'il avait reçu au sujet de cette grotte sans même qu'il en eut conscience. Baignant dans la connaissance Iliian, Isaac n'avait reçu que l'emplacement de ce qu'il avait rapidement nommé un hangar, décidant que la présence de vaisseaux ne pouvait signifier qu'un gigantesque spatioport emplit de navettes et de techniciens affairés. Amusé de sa puérilité, il s'identifia un peu plus à la communauté Iliian et la raison qui le poussait à prendre leur flotte de vitesse, une grenade à son bord et soutenu par l'espoir que son corps imitait bien les propriétés des plantes de la planète aux Iles s'affaiblit brutalement. Isaac resta immobile quelques minutes, fixant peu à peu son identité et ses pensées. Puis il alla chercher la grenade qu'il avait déposée à l'entrée de la grotte et la laissa tomber dans le bassin épais situé au fond du réduit. Il s'assit ensuite à son bord et tendit le bras pour toucher le filament dont il pouvait apercevoir les racines, étalées à la manière de veines sous les parois du bassin, palpitant à cette cadence brisée que suivait désormais chaque cellule d'Isaac.


Le filament s'étendit instantanément, s'enroula autour de son bras, décrivit une spirale sur sa poitrine et disparut non loin de la zone où se serait trouvé un plexus solaire si Isaac en avait possédé un. Il ressentit une traction et tomba tout entier dans le gel, bloquant sa respiration et entrant en position fœtale ; traces d'humanité ressurgissant une dernière fois avant de disparaître à jamais.


Les parois du bassin se contractèrent au contact de sa peau et les filaments s'échappèrent de leur gangue pour venir à sa rencontre. A l'extérieur, les abords se resserraient doucement, laissant apparaître la cavité rocheuse dans laquelle le bassin avait été logé. Le corps d'Isaac consommait la matière environnante, la transformait à mesure que son génome mutait. Par vagues successives, sa masse nerveuse coula le long et son crâne et s'étala le long de son corps à présent informe. Les structures neurales étaient dupliquées, stockées de façon redondante au gré des fluctuations que ses autres chairs subissaient au même moment. Une pensée jaillit et retomba aussitôt dans le magma de présences qui se reconstruisait peu à peu en un individu unique.


\s


Lorsqu'Isaac s'éveilla, il ne put ouvrir les yeux. Il accéda alors à la conscience Iliian dans l'espoir de trouver des informations quand à son état. Il avait pressenti que la grotte sous marine n'abritait rien d'autre que le moyen de devenir l'une de ces navettes rapides dont il avait besoin. Néanmoins, dépourvu d'yeux, de bras, et de jambes, il était pressé de reprendre le contrôle de son organisme, inquiet d'être, sans le savoir, dans un environnement mortel pour son nouveau corps. Il apprit que sa structure cérébrale avait été répartie au travers de son organisme afin de protéger l'individualité lors d'un crash éventuel et il ne put obtenir d'image mentale de la forme de son nouveau corps. La seule pensée Iliian qui put lui venir en aide était celle d'un de ses congénères, lui aussi sous forme de navette. S'ouvrant à ses perceptions, il apprit la nature de ses nouveaux sens et la façon d'en user.


La lumière et le son jouaient un rôle secondaire dans le nouvel univers d'Isaac. Il pouvait en revanche percevoir d'autres fréquences électromagnétiques et reconnaissait finalement chaque champ gravitationnel influençant celui où il se trouvait à un instant donné. Il ne put trouver d'équivalents humains à la méthode qu'il utilisait pour se déplacer. Mais dans ce monde, les vitesses ne semblaient pas tant augmentées que les distances n'étaient réduites. Le système solaire était ridiculement proche. Il s'y propulsa.


Quelque part dans les tréfonds de son organisme se logeait toujours la grenade qu'il avait laissé tombé dans le bassin où il avait plongé à son tour.

\s


Isaac traversa les immensités célestes. Suivant les courbes gravitationnelles, il passa par de nombreux endroits riches en espèces dont il pouvait ressentir la présence. Maintes fois, il regretta de n'avoir pas d'yeux pour admirer les lieux où il se trouvait. Mais au-dessus de chaque pensée, son but surnageait et le guidait dans ses mouvements. Il préférait tuer l'humanité des hommes plutôt que de tuer les hommes eux-mêmes, et ce avant qu'il ne fut lui-même trop Iliian pour ne plus se soucier d'une pareille distinction.

\s

La Terre approchait. Les heures étaient nombreuses avant que ne surgisse la flotte Iliian, aussi Isaac en profita pour s'assurer de la présence d'un grand nombre d'humains à la surface de la planète. Utilisant pour la première fois l'un des innombrables organes dont il était pourvu, chacun destiné à remplir une fonction aussi précise que mystérieuse, il tenta d'intervenir sur les rares gaz l'environnant. Changeant leur phase, il parvint à générer un plasma dont il orienta le flux vers les pôles magnétiques de la Terre. Réagissant avec la haute atmosphère, ceux-ci recréèrent une aurore boréale d'une étendue et d'une puissance suffisante pour être visible en plein jour jusqu'au niveau de l'équateur.


Mais l'astre était peuplé par ses ennemis. Il ne savait plus pourquoi il les tuait lui-même. Ses frères s'apprêtaient à effectuer quasiment le même travail.


Non.


Il y avait une différence. Quelque chose dont lui seul était capable. Il était en train de sauver ses semblables. Non, ses semblables étaient dans son dos, sur le point d'éliminer le cancer humain. Mais face à lui, pourtant, vivaient les siens. Il fallait les éliminer, eux et l'innommable chaos qu'ils incarnaient.


La sérénité en Isaac devint une entité à part entière. Elle se tourna lentement vers lui, et son regard dédaigneux porté par deux yeux infinis l'aspira, lentement, loin de son mouvement intérieur aussi repoussant que les créatures qui peuplaient cette planète immonde. Le mouvement emplit le regard. Il happa sa profondeur infinie de l'intérieur et la contint au creux de son agitation éternelle. Isaac revint à lui.


Conscient du peu de temps dont il disposait avant que sa conscience ne meure à jamais, il agit mécaniquement, offrant peu de place à sa pensée pour le détourner de son but. Déclenchant ce qui lui parut être un frisson volontaire, il détacha tous les filaments dont son corps était couvert. Les appendices de ce composé organique destiné à fournir aux Iliians la capacité de voyager dans l'espace se rétractèrent jusqu'à leur cœur, qui se détacha d'Isaac et dériva au loin. La matière excédentaire se détachait peu à peu de lui, sculptant à nouveau sa forme Iliian. Quelque part au milieu de cette gangue se trouvait sa grenade. Isaac battit des bras, rua en tous sens, et finit par la heurter. Il s'en saisit, et attendit.


Bientôt il ne fut plus qu'un Iliian seul au-dessus de la Terre, vide de toute ressource et condamné à s'éteindre lentement, bien trop tard pour ne pas être brûlé vif par l'atmosphère qu'il approchait déjà.


Isaac dégoupilla alors la grenade et régla sa puissance au maximum. Il se lova autour d'elle.


Lorsque la grenade explosa, des millions de particules jaillirent dans l'espace. Chaque lambeau de sa chair suivit une orbite différente et leur entrée dans l'atmosphère eut lieu au-dessus de toute la surface de la planète. Plus de la moitié disparut durant la chute, carbonisée par le frottement contre les gaz. Mais quelques éléments de la résistante structure Iliian tinrent bon. Portés par le vent, des cellules flottèrent entre les hommes. Chaque brin d'ADN en elles était unique. Le génome Iliian modifié par la maladie de l'humain qui s'était transformé en l'un d'eux était devenu contagieux. Lorsque les cellules touchèrent des corps humains tirés de leurs abris par le spectacle céleste, ceux-ci s'effondrèrent d'effroi devant le spectacle d'écailles surgissant à la surface de leur peau.


Lorsque la flotte Iliian arriva à destination, elle reçut un message. Leurs frères les attendaient.

\end{document}
